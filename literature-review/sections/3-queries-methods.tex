\section{Search Queries}

The related work section research started with 7 recommended papers by my Supervisor. Based on the keywords and related work of those papers a systematic approach was adopted, employing a combination of specific search queries tailored to the research focus. The academic database from the UvA, Catalogueplus, was used with the following queries:

\begin{itemize}
    \item "Human-Building Interaction" OR "HBI" AND "Built Environments"
    \item "Indoor Air Quality" OR "IAQ"
    \item "Post-occupancy Evaluation" OR "POE"
    \item "Indoor Environmental Quality" OR "IEQ"
    \item "Data Physicalization" OR "DataPhys" OR "Tangible Data"
    \item "Ubiquitous Computing" OR "Ubicomp" AND "Persuasive Technology" OR "Behavior Change"
    \item "Sick Building Syndrome" OR "SBS" AND "Respiratory Comfort" OR "Cognitive Functions"
    \item "Thermal Comfort" OR "Acoustic Comfort" AND "Indoor Occupant Comfort" OR "Comfort Dimensions"
\end{itemize}

In catalogueplus I filtered on peer reviewed and narrowed the scope to the last 5 years. Some key (state of the art papers) where part of special issues of specific journals so what I did was then look into the latest issue to see most recent studies. Since HCI is the main field of focus key papers are usually in the ACM library in my field of study. So whenever I found a paper I also refined my queries within the ACM library.

\section{Methods}

For the topics in the introduction and research questions I created folders in Zotero and using the Zotero connector plugin in my browser you can pin papers in the right folders. Used the sorting and labelling features to assign topics to them within the Zotero app. Zotero integrates with the beta of a tool called Research Rabbit which creates a knowledge graph (visual) of papers in your zotero collection. It cross-references the papers and give you similar work in the field based on relevance so you can discover subfield and author networks.
