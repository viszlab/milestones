\section{Search Queries}

The exploration of relevant literature started with seven seminal papers recommended by my supervisor. Based on the keywords and related work of those papers a systematic approach was adopted, employing a combination of specific search queries tailored to the research focus. The academic database from the UvA, University Library Catalogueplus \footnote{https://lib.uva.nl/discovery/}, was used with the following queries:

\begin{itemize}
    \item "Human-Building Interaction" OR "HBI" AND "Built Environments"
    \item "Indoor Air Quality" OR "IAQ"
    \item "Post-occupancy Evaluation" OR "POE"
    \item "Indoor Environmental Quality" OR "IEQ"
    \item "Data Physicalization" OR "DataPhys" OR "Tangible Data"
    \item "Ubiquitous Computing" OR "Ubicomp" AND "Persuasive Technology" OR "Behavior Change"
    \item "Sick Building Syndrome" OR "SBS" AND "Respiratory Comfort" OR "Cognitive Functions"
    \item "Thermal Comfort" OR "Acoustic Comfort" AND "Indoor Occupant Comfort" OR "Comfort Dimensions"
\end{itemize}

In CataloguePlus, filters were utilized to aim for peer-reviewed articles published within the last 5 years (2019 - 2024). Notably, several pivotal papers, constituting the state of the art, were featured in special issues of particular journals. Consequently, the latest issues of these journals were reviewed to access the most recent studies within the subfield. Given the central focus on Human-Computer Interaction (HCI), key papers were frequently found within the ACM library \footnote{https://dl.acm.org/}. Consequently, searches within this repository were systematically refined upon the identification of relevant literature.

\section{Methods}

The related work section does not serve as a conventional background section; rather, it predominantly highlights the state-of-the-art advancements within the research fields, with minimal coverage of foundational literature. It assumes a scientific audience with a preliminary knowledge of the researched subfields. Each subsection includes a paragraph outlining the differences between this research and prior works, thereby establishing the specific research gap addressed in the study.

To organize the topics introduced in the thesis and delineate the associated research questions the citation management tool Zotero \footnote{https://www.zotero.org/} was used, and specific subfolders were created to accommodate these topics which translated to the subsections outlined in the related work draft. Through the integration of the Zotero connector plugin into web browsers, relevant scholarly papers could be conveniently pinned to the appropriate folders. Subsequently, within the Zotero application itself, sorting and labeling functionalities were employed to categorize the pinned papers according to their respective topics. Furthermore, the integration of Zotero with the beta version of Research Rabbit \footnote{https://www.researchrabbit.ai/}, a tool designed to construct visual knowledge graphs of papers contained within Zotero collections, was used. This integration facilitated cross-referencing of papers and provided recommendations for related works based on similar works and references within the paper thereby enabling the exploration of subfields and author networks.
