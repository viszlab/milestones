\section{Metadata}

\begin{itemize}
  \item \textbf{Stakeholder:} University of Amsterdam
  \item \textbf{Research group:} Digital Interactions Lab (DIL)
  \item \textbf{Supervisor:} dr. Hamed S. Alavi PhD (h.alavi@uva.nl)

\end{itemize}

\section{Project Brief}

\textit{The broad objective of this project is to analyze and visualize data on the comfort levels at LAB42 building. This should include the four classic dimensions of comfort (thermal, visual, acoustic, and air quality), but also other aspects such as visual privacy, accessibility, and so forth. Data-oriented methods will be applied to elicit the most influential factors that determine the overall sense of comfort in the LAB42 building at the individual as well as social levels.}

\section{Introduction}

The focus of the project is broadly how humans perceive, experience and interact with environments. In this case particularly the Lab42 building at the University of Amsterdam. Currently, there is already a lot of data being collected (e.g. based on the four classic dimensions of comfort thermal, visual, acoustic, and air quality) and exposed through several APIs.

\section{Objectives \& Goals}

Working on this project for me personally would evolve around two core concepts that interest me and I would like to explore further:

\begin{itemize}
  \item Quantifying and collecting data about human awareness of the environment that isn't easily 'measured' using sensory data (e.g. privacy, accessibility, social awareness) since they are often subjective and unique for each individual.
  \item Dynamically displaying gathered (sensory) data using physical hardware that integrates and blends into the environment. In the form of a physical (often called 'tangible') real-life data visualization.
\end{itemize}

\section{Focus areas}

My current focus in terms of skills and projects is mostly on human-centered design, ubiquitous computing, sensory data, and in general hardware and microcontrollers. Next to my experience (based on my Bachelor's and work experience) as a web developer I worked at several agencies as a front-end developer and back-end developer often working on data visualization projects. This allows me to have knowledge of both hardware to make and prototype tangible products but also integrate in terms of software such as accessing API data, setting up back-end environments and creating front-end websites.

In terms of focus areas for the project I'm particularly interested in;

\begin{itemize}
  \item Devices that offer multi-modal interaction (MMI) and are thus multi-modal systems (MMS), such as auditive feedback, tactile feedback between humans and computers (HCI)
  \item Devices that have a clear feedback loop with a cue, affordance, feedback, feedforward and prompts (multi-modal design).
  \item Devices that focus on the change of human behavior such as persuasion where humans are influenced or convinced to take specific actions, make certain decisions, or adopt particular behavior (persuasive HCI).
\end{itemize}

\section{Possible deliverables}

Most likely prototypes of a physical tangible (art / screenless) artifact. In an ideal case, the artifact should allow for ways of interacting with humans to offer meaningful human-building interaction (HBI). Probably including a 'hardware' component, in the form of a microcontroller, sensors, haptic motors, AI modules (e.g. vision, speech) and a 'material' component that uses materials to create a tangible artifact user might interact with (e.g. 3D printing, laser cutting, vacuum forming).