\documentclass[a4paper]{article} 

% Setup packages
\usepackage{booktabs}
\usepackage{graphicx}
\usepackage{multicol}
\usepackage[T1]{fontenc}
\usepackage[utf8]{inputenc}
\usepackage{fancyhdr}
\usepackage{float}
\usepackage{helvet}

\graphicspath{/}

\renewcommand{\familydefault}{\sfdefault}

\addtolength{\hoffset}{-2.25cm}
\addtolength{\textwidth}{4.5cm}
\addtolength{\voffset}{-3.25cm}
\addtolength{\textheight}{5cm}
\setlength{\parskip}{0pt}
\setlength{\parindent}{0in}

% Set the path for media
\graphicspath{ {media/} } 
\pagenumbering{arabic}

\begin{document}

\includegraphics[scale=0.2]{uva-logo.jpg} \\

\section*{Ethics Application}

Ethics Application Submission for the Information Sciences Ethics Committee (ECIS) in accordance with \emph{Appendix II} of the \emph{Handbook Ethics v1.2}. This application has been forwarded to the secretary of the ECIS committee for evaluation, and submitted through the online submission form. \\

In addition to completing this project-specific questionnaire from the handbook, this research, in general, adheres to the UvA Privacy Statement \footnote{https://www.uva.nl/en/home/disclaimers/privacy.html} and the UvA code of conduct \footnote{https://www.uva.nl/en/about-the-uva/policy-and-regulations/codes-of-conduct-and-social-safety/code-of-conduct-of-the-uva.html} when interacting with individual users. The researcher commits to upholding the ethical guidelines outlined in these documents. \\

This application represents the researcher's current understanding at the outset of the thesis research. As the project progresses, collection methods and data may undergo slight modifications. In the event of significant changes, the Ethics Committee will be promptly informed.


\section*{Applicant Information}

\begin{description}
  \item \textbf{Name:} BSc Danny de Vries (14495643)
  \item \textbf{Email:} danny.de.vries@student.uva.nl
  \item \textbf{Project:} Master Thesis (IS)
  \item \textbf{University:} University of Amsterdam (UvA)
  \item \textbf{Master:} Information Studies Information Systems (track)
  \item \textbf{Institute:} Informatics Institute
  \item \textbf{Faculty:} Faculty of Science (FNWI)
  \item \textbf{Research Group:} Digital Interactions Lab (DIL)
  \item \textbf{Supervisor(s):} Dr. H. (Hamed) Seiied Alavi PhD \& Shruti Rao Ph.D. Candidate
\end{description}

\section*{Questionnaire}

\section{Generic questions}

\subsubsection{Title of the project}

A breath of fresh air: towards optimal Indoor Air Quality (IAQ)

\subsubsection{Aim of the research}

This master thesis investigates the integration of sensory measurements of indoor air quality (IAQ) into specific indoor spaces to raise occupants’ awareness and help occupants take preventive measures against poor air quality. Conducted at Lab42 the study employs human-building interaction principles and persuasive technology to visualize IAQ data. Key research questions address methods for collecting accurate IAQ measurements, integrating environmental data into visual representations, and evaluating the impact of physical visualizations on occupants’ understanding and behavior. The overall aim is to inform the development of effective interventions, contribute to healthier indoor environments, and inform building design decisions.

\subsubsection{Short description of the project}

Integration of user studies throughout the entire process, employing a systematic approach to data collection, and prototyping design solutions are the main focuses of this research. The chosen methodologies are project-oriented and form part of in-the-wild studies aimed at examining occupant behavior. These studies are complemented by data collected from sensing devices which all go through a process of data cleaning, transformation, and analysis. Prototyping of sensing devices, persuasive technology, and data physicalizations will be developed and subjected to usability testing to assess potential changes in occupant behavior before and after installation. The overarching goal is to inform the design through data, utilizing design as a probing tool for data collection and vice versa. This approach ensures a comprehensive understanding of user interactions and space dynamics, ultimately contributing to a design solution that is both informed by empirical data and responsive to occupant needs.

\subsubsection{Principle investigator}

Danny de Vries

\subsubsection{Conducting researchers}

Danny de Vries

\subsubsection{Co-Applicants for this application}

Dr. H. (Hamed) Seiied Alavi PhD \& Shruti Rao Ph.D. Candidate

\subsubsection{Sub-department}

Informatics Institute Faculty of Science (FNWI)

\subsubsection{Names of organisations involved}

Digital Interactions Lab (DIL)

\subsubsection{Location of project execution}

Lab 42 building at the UvA Science Park Campus

\subsubsection{Expected research duration}

1th of February 2024 until 30th of June 2024

\subsubsection{Have you proposed this or a similar project to the ECIS in the past?}
No

\subsubsection{Will external objects be involved in the project?}
Existing datalogger and sensing devices, used in previously approved studies, from the Digital Interactions Lab will be used. Furthermore, there is a potential for the development of a custom sensing device to augment the existing data collection process.

\subsubsection{Is the owner of the object aware of this project?}
Yes

\subsubsection{Is the project likely to attract attention from the media?}
No, at the time of writing the researcher does not foresee the project reaching large public media attention.

\section{Data collection}

\subsubsection{What data will you be collecting, generating, reusing?}

There are 'three' components during the research that will gather data in some form: \\

1) Space measurements (sensing devices): Analyzing current space with regards to air quality by utilizing the data measured from the automated systems already installed within the building exposed by the building API and augmenting this dataset with commercially available indoor climate data loggers from Atal. Concurrently, observations of occupants will be conducted to cross-reference the data with their actual behavior. \\\\
2) Occupant behavior (elicitation study): Collecting information about occupants within the specific spaces within the building will involve assessing their comfort states. Surveys are anticipated to be the primary data collection method distributed across the course of the thesis projects. Additionally, conducting one-on-one interviews using open-field questionnaires to explore occupants’ comfort levels more comprehensively will also be used.\\\\
3) Design solution (prototyping): After gathering data and some preliminary occupant studies the aim is to prototype a design solution for behavior change, emphasizing the concept of calm technology—prompting occupants to undertake preventive actions with minimized interruption costs. Most likely in the form of a tangible visualization of the air quality within a room. The designed solution will undergo usability testing and subsequent data analysis, enabling a comparative evaluation of user behavior both before and after installation. \\

All occupant data is anonymized and nothing that can be considered personal data is collected. Data such as full name, contact information, identification numbers, date of birth, health information, biometric data, financial information, demographic, and employment information will not be collected during the research.

\subsubsection{How much data do you plan to collect?}

Text files (.pdfs of surveys, interviews, and field observation), Spreadsheets (.csv of surveys and interviews), Images of the building, prototypes, and sensing devices (.jpeg, rasterized images) Sensory data (.json objects). In terms of actual file size somewhere between 0 - 2 GB.

\subsubsection{Where will the data be stored during the research project?}

Data will be permanently stored on cloud service providers provided by the University of Amsterdam (such as Research Drive and Surfdrive) as outlined in the Research Data Management guidelines \footnote{https://rdm.uva.nl/en/looking-after/storage/storage.html}. 

\subsubsection{How will the data be organized?}
Sensory data will be labeled and timestamped when sensing devices are installed. All other files will be organized within subfolders and tagged within the cloud storage.

\subsubsection{What are the participant/author’s expectations of privacy?}
Information about the project, what data will be collected, and how the data will be used will be communicated before the participants agree to partake in the research. Survey consent forms are included when users are filling in the survey. They are welcome to opt-out at any moment. The same goes for the more in-depth interviews where users are given a consent form before agreeing to the interview. Possible sensory devices and prototypes do not gather any personal information that can be traced back to individual users and users within the space are notified of the sensory devices when they enter the space and building.

\subsubsection{Does the service’s privacy policy contradict ethical principles?}

For surveys and interviews, Qualtrics will be used. The privacy policy of Qualtrics doesn't violate the privacy principles outlined in the UvA privacy statement. Files in SURFdrive and Research Drive are GDPR compliant and do not leave the Netherlands. 

\subsubsection{What measures safeguard data at the site of data collection?}

Sensor devices and prototypes will store data locally (no 'over-the-air' data transmission) and will be collected daily. After collection, the memory and storage of these devices will be wiped (reformatted). During analysis, the data will be locally stored on encrypted laptops (e.g. password-protected Macbooks with FileVault enabled). 

\subsubsection{How will images/audio be effectively anonymized?}

If images are taken for research purposes (e.g. observation study) within the buildings all visible faces will be post-processed with a blur to cover faces if published. Any audio recorded (e.g. interviews) will be post-processed to remove any references to personal data if published. It is not likely these input files and data will be distributed and published, in principle they will only be used for analysis and viewed by the researchers working on the project.

\subsubsection{How is profile or location information used or stored by the researcher?}

Interviews and surveys might include some form of location data within the building (e.g. roughly a description of the place where they are located within the building). Since the users are anonymized and no personal information will be collected location data can not be traced back to individual users. Location data within spaces of the building are only used for collective monitoring and aggregative data analysis.

\subsubsection{Is the author/subject a minor?}

It is expected that most users entering the designated research areas in the case study building are not minors. It might be possible minors may be approached to participate in the study since some individuals within the focus group of the building (students) will most likely be under the age of 18 years but students that are minors are not the focus group of the research.

\section{Data security}

\subsubsection{What are the main risks to data security?}

Unauthorized access (by others than the researchers) in general is the main risk to data security. 

\subsubsection{Who will have access to the data?}

Only researchers in possession of the encryption methods and passwords will be able to retrieve the data from the sensors. Data from the user studies from the cloud storage will only be accessible to the researcher and supervisor(s).

\subsubsection{How will the data be destroyed?}

The data will be removed, after collection of the data, by formatting the storage of the sensory devices. The user data will remain in the cloud storage while the thesis research is ongoing. The data will be accessible for a certain period after the study has been concluded and then deleted after graduation.

\subsubsection{Are the activities in conflict with regulations?}

Not particularly, installing the devices and possible prototypes will be discussed and application for approval will be done to faculty staff and building administrators.

\subsubsection{Which law applies?}

In general, the Law of the Netherlands. Specifically for this research any privacy and storage guidelines by the University of Amsterdam, the  Dutch Data Protection Authority (AVG), and the European General Data Protection Regulation (GDPR).

\section{Data archival and preservation}

\subsubsection{Describe what data will be archived after the end of the project, and how?}

All collected user study data and sensory data will be archived as described earlier, on the cloud service in the form of text files (.pdfs of surveys, interviews, and field observation), spreadsheets (.csv of surveys), data log objects (.json objects).

\subsubsection{What is the retention period of data?}

The user data will remain in the cloud storage while the thesis research is ongoing. The data will be accessible for a certain period after the study has been concluded and then deleted after graduation.

\subsubsection{Where will the data be archived?}

For long-term storage and archival purposes, user study and sensory data data will be archived in external hard disks stored in a locked office which is only accessible to the researchers involved in the project after the project has finished. The default retention period is 2 years.

\section{Data publication and access}

\subsubsection{What data will be published as outputs from the project and when?}

Upon completion of the master thesis aggregation of the analysis will be published in the results section of the thesis paper. Notebooks of the analysis performed will be published in a public GitHub repository as well as the source code of the sensory devices and prototype. The data used for these analyses will not be published on GitHub and will be kept private in the cloud storage and on local devices.

\subsubsection{How long will the data exist in the repository?}

The GitHub repository of the output analysis will remain published and public during the research and will remain on GitHub after completion of the master thesis project. Contact information will be supplied in the documentation (e.g. readme.md) on how to reach out to the researchers for possible requests for code removal.

\subsubsection{What consent is needed for subsequent data use?}

The outputs of the analysis and the code of the prototype are hosted on GitHub and subsequent use or references of these analyses must adhere to the licenses within the repository. If the input data for future research needs to be acquired a request can be made to the researchers and will be reviewed for approval.

\section{Data protection}

\subsubsection{During the research process, what sensitive data relating to subjects are processed?}

All user data is anonymized and nothing that can be considered personal data is collected. Data such as full name, contact information, identification numbers, date of birth, health information, biometric data, financial information, demographic, and employment information will not be collected during the research.

\subsubsection{Does the remixing/mashing of data enable the identification of individuals?}

Not to the knowledge of the researchers. As stated above, no personal data is being collected and also no location-specific data such as GPS coordinates or IP addresses are collected.

\subsubsection{Is informed consent given by participants for this research?}

Yes, consent forms will be handed out before user study methods are employed. Users can opt out of the research at any moment.

\end{document}