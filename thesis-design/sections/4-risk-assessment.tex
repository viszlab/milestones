\section{Risk Assessment}

This research approach is dependent on in-the-wild research and placing and testing several devices throughout an actively used building, this proposes certain constraints that are discussed in this section.

\subsection{User studies}

While the research is not entirely dependent on the available existing data, incorporating survey data and interviews with users is crucial for comprehensive insights. In the event of limited survey responses or challenges in conducting interviews, there is a potential risk of information saturation. However, given the research timeline and the building's visitor frequency, this scenario is considered unlikely.

\subsection{Building data}

There is the possibility that access to data building is limited to gathering significant data about space behavior. It is reported that not all sensors listed in the overview are confirmed to be active and running. This can be mitigated by enhancing the already existing sensors with prototype sensing devices to gather data as a proof of concept. Also, only relying on data is a risk factor and incorrect processing of the data can lead to incorrect evaluation of space behavior.

\subsection{Building access}

There is a possibility of restricted access to the building due to construction safety or administrative reasons, hindering the testing of the design solution and placement of sensing devices. Coordination with building faculty staff is essential to address this potential constraint. While no objections have been raised in prior research, the risk can be mitigated by testing the prototype in an alternative context to assess device usability.

\subsection{Privacy and ethical considerations}
Sensing and gathering sensory data from users is data collection that users might deem as privacy-invasive. Careful considerations should be made to mitigate these concerns and users should be confident that data for this research is gathered anonymously and only for collective monitoring. Interacting with users within the building will be in alignment with the principles outlined in the UvA code of conduct \footnote{https://www.uva.nl/en/about-the-uva/policy-and-regulations/} and an application to the Ethics Committee for Information Sciences (ECIS) \footnote{https://ivi.uva.nl/research/ethical-code/ethical-code.html} about how data is being stored and gathered has been made. Advice from the committee is still pending.