\section{Risk Assessment}

Although this research is not entirely dependent on the available existing data, since part of the research is to gather data by the researcher, the results will most likely benefit from access and analyzing existing user and building datasets.

\subsection{Interview and surveys}

If no or not enough interviews can be conducted due to time constraints or unavailable interviewees, there will be a lack of information which leads to an absence of information saturation. 

\subsection{Building sensory data}

Access to data building is not properly exposed or current building data is limited to gather significant data about occupancy. Which means gathering data about the current building is limited. This can be mitagated by enhancing the already existing sensors with prototype sensing devices to gather data as a proof of concept. Only relying on data is a risk factory and icorrect processing of the data can lead to incorrect evaluation of user behaviour. The data should thus always be cross-referenced with survey data. However, given that it’s a new building, it is reported that not all sensors listed in the overview confirmed to be active and running.

\subsection{Installation}

Due to construction or administrative reasons it might not be possible to test the eventual design solution in the building at scale. This needs to be discussed with building faculty staff. This can be mitigated by testing the prototype in a different context to test it's usability.

\subsection{Privacy considerations}
Sensing and gathering sensory data from users is data collection that users might deem as privacy-invasive. Careful considerations should be made to mitigate these concert and be open and transparent to users what data is being collected and how it's beind processed. Users should be confident that data for this research is gather anonysmouly and only for collective monitoring, so they should not be traced back to individuals users and location.

\subsection{Ethical considerations}

Since most of this research involves user studies. The data requires the researcher
to act with great care, taking appropriate precautions the data is only examined on site within the constraints of the building and UvA faculty. Interacting with users within the building will be confirmed following the code of conduct for the HvA and an application to the ECIS about how data is being stored and gathered has been made. An advice from the committee is still pending.