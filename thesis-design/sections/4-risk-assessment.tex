\section{Risk Assessment}

This research approach is dependent on in-the-wild research and placing and testing several devices throughout an actively used building, this proposes certain constraints that are discussed in this section.

\subsection{Occupant behavior studies}

While the research is not entirely dependent on the available existing data, incorporating survey data and interviews with occupants is crucial for comprehensive insights. In the event of limited survey responses or challenges in conducting interviews, there is a potential risk of information saturation. However, given the research timeline and the building's visitor frequency, this scenario is considered unlikely.

\subsection{Air quality and building data}

There is the possibility that only relying on access to building data is limited to gathering significant data about comfort and air quality. It is reported that not all sensors listed in the overview are confirmed to be active and running. This can be mitigated by enhancing the already existing sensors by installing dataloggers in designated areas to gather data besides the already existing sensors present in the building.

\subsection{Building access and hardware installation}

There is a possibility of restricted access to the building due to construction safety or administrative reasons, hindering the testing of the design solution and placement of sensing devices. Coordination with building faculty staff is essential to address this potential constraint. It needs to be noted that no objections have been raised in prior research, but this risk can be mitigated by testing the prototype in an alternative context to assess device usability.

\subsection{Privacy and ethical considerations}
Sensing and gathering sensory data from occupants is data collection that occupants might deem as privacy-invasive. Careful considerations should be made to mitigate these concerns and occupants should be confident that data for this research is gathered anonymously and only for collective monitoring. Interacting with occupants within the building will be in alignment with the principles outlined in the UvA code of conduct \footnote{https://www.uva.nl/en/about-the-uva/policy-and-regulations/} and an application to the Ethics Committee for Information Sciences (ECIS) \footnote{https://ivi.uva.nl/research/ethical-code/ethical-code.html} about how data is being stored and gathered has been made. Advice from the committee is still pending.