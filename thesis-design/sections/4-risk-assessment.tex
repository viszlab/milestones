\section{Risk Assessment}

This research approach is dependent on in-the-wild research and placing and testing several devices throughout an actively used building, this proposes some constraints that are discussed in this section.

\subsection{User studies}

Although this research is not entirely dependent on the available existing data, since part of the research is to gather data, the results will most likely benefit from accessing and analyzing existing users. If survey data is no or not enough interviews can be conducted due to time constraints or unavailable interviewees, there will be a lack of information which leads to an absence of information saturation. However, this is highly unlikely due to the period of this research and amount of visitors to the building.

\subsection{Building data}

There is the possibility that access to data building is limited or current building data is limited to gather significant data about user behavior. It is reported that not all sensors listed in the overview are confirmed to be active and running. This can be mitigated by enhancing the already existing sensors with prototype sensing devices to gather data as a proof of concept. Also, only relying on data is a risk factor and incorrect processing of the data can lead to incorrect evaluation of user behavior.

\subsection{Building access}

Due to construction safety or other administrative reasons, it might not be possible to test the eventual design solution in the building at scale and place the sensing devices to gather additional data. This needs to be properly discussed with the building faculty staff. It needs to be noted that no prior objections for previous research were raised and this constraint can be mitigated by testing the prototype in a different context to still test the usability of the devices.

\subsection{Privacy and ethical considerations}
Sensing and gathering sensory data from users is data collection that users might deem as privacy-invasive. Careful considerations should be made to mitigate these concerns and users should be confident that data for this research is gathered anonymously and only for collective monitoring. Interacting with users within the building will be in alignment with the principles outlined in the UvA code of conduct and an application to the Ethics Committee for Information Sciences (ECIS) about how data is being stored and gathered has been made. Advice from the committee is still pending.