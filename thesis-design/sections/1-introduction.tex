\section{Introduction}

This thesis will investigate [...] 

\subsection{Research questions}

In order to achieve this, the following main research question is formulated: \\

\emph{How can persuasive technology combining user data and ubiquitous computing design impact user behavior to significantly enhance the quality of informal learning spaces to facilitate an optimal learning environment?} \\

To be able to answer this research question the following supporting sub-questions, which also act as objectives to describe which knowledge needs to be acquired to be able to answer the main reserach question, are formulated:

\begin{itemize}
    \item \textbf{RQ1:} \emph{What are the characteristics, intentions and goals of the users entering the building? (descriptive)}
    \item \textbf{RQ2:} \emph{How do users of the building currently define and rate their personal comfort in relation to the building? (defining)}
    \item \textbf{RQ3:} \emph{What sensory data about users and the environment is currently being collected in the building and can this be enhanced? (defining)}
    \item \textbf{RQ4:} \emph{How can a ubiquitous computing device (persuasive technology) nudge users into certain desired behavior? (designing)}
    \item \textbf{RQ5:} \emph{Is there a difference in user-behaviour pre-installation and post-installation of the device?}
\end{itemize}

As outlined further in the \emph{related work} section research has been done on smart buildings, the field of ubiquitous computing and sensory data has also seen a lot of research especially with the advancement of low-cost hardware components. Less research has been done to give a definition comfort within indoor buildings and many of this research is limited to gathering data and analyzing but not much research is performed on actually measuring user behaviour and if with ambient devices users can be nudged or pushed to take preventive action and if components within a building can be adapter to and change behaviour especially within the niche and focus of this case study; students as a focus group in the context of informal learning environments.

\subsection{Problem statement}

Buildings are more and more getting equiped with sensors and automated systems that regulate based on general parameters set by policymembers and administrative staff. Users within buildings have limited control to make adjust in terms of their comfort. In short: \emph{the building must adapt to the user instead of the user adapting to the building.} This research specifically helps to efficiently arrange the Lab-42 building and helps faculty staff to make decisions on how to optimally arrange spaces but in general can inform architecture and builders to optimize further university buildings with a focus on informal learning spaces.

\subsection{Lab42 building}

This research will be performed in association with the \emph{Digital Interactions Lab} and uses the recently (september 2022) opened Lab42 \footnote{https://lab42.uva.nl/} building at the UvA Amsterdam Science Park as a case study. Lab42 is a energy-neutral, flexible and adaptable designed faculty building that facilites partnerships between students, researchers and businesses. \cite{crouwel} The layout aims to feature different zones with varying functionalities, from areas where you can sit quietly and focus on work to spaces that allows for collaborative work. The overarching interior theme in the design is 'tech' and 'nature' aiming to create afresh, light and warm comfortable building. Sensing devices are installed throughout the building to automatically adjust lighting, air, temperature so these can be adjusted for overall improvement of comfort \cite{faculty}.