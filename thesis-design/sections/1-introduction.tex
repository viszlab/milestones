\section{Introduction}
Globally, it is estimated that individuals typically allocate 90\% of their time within various indoor environments \cite{indoor}. New (smart) buildings are often retrofitted with sensors and automated systems that aim to regulate comfort in a multi-dimensional approach encapsulated in thermal, respiratory, visual, and acoustic dimensions. 

One of these dimensions that can have far-reaching implications for occupants' comfort and health is the quality of indoor air (IAQ). Insufficient ventilation in indoor spaces leads to poor air quality and an increased occurrence of discomfort and diminished well-being experienced by occupants \cite{ventilation}. The essence of air quality is its invisibility to occupants; polluted air is not easily detected by smell or sight. Additionally, mechanical ventilation systems in buildings operate discreetly, contributing to occupants' perceived lack of control. These systems are typically automated and cannot be directly regulated or controlled by occupants themselves.

This thesis focuses on understanding occupants' needs through in-the-wild studies measuring indoor air quality in specific spaces (e.g. meeting rooms) and prototyping various persuasive technologies and data physicalization devices to visualize indoor air quality and evaluating their effectiveness with the overall goal of gathering insights into occupants' comfort levels and helping them to take preventive action against poor indoor air quality. This creates an interplay between occupants' health and comfort, architecture and built environments, and computing technologies.





\subsection{Research questions}

In order to research the enhancement of informal learning spaces through ubiquitous computing:  the following main research question is formulated: \\

\emph{How can ubiquitous computing impact student behavior to significantly enhance the quality of informal learning spaces and facilitate an optimal learning environment?} \\

To effectively answer this main research question, this research is guided by the following supporting sub-questions that also serve as objectives to delineate the necessary knowledge: \\

\begin{itemize}
    \item \textbf{RQ1:} \emph{What are the characteristics, intentions, and goals of the students entering the building? }
    \item \textbf{RQ2:} \emph{How do students of the building currently define and rate their comfort concerning subjective parameters of the building? }
    \item \textbf{RQ3:} \emph{What sensory data about students and the environment is currently being collected and should this be enhanced? }
    \item \textbf{RQ4:} \emph{How can ubiquitous computing devices nudge students into certain desired behavior? }
    \item \textbf{RQ5:} \emph{Is there a difference in user behavior pre-installation and post-installation of the computing devices? }\\
\end{itemize}

As outlined in the \emph{related work} existing research has explored human-building interaction, the field of persuasive technology, data physicalization, and informal learning spaces. While research on defining comfort within indoor buildings and gathering and analyzing sensory data is prevalent, there is a gap in understanding user behavior and their subjective needs. Moreover, limited research exists on how design solutions can empower users, providing them with control and enabling spaces (including buildings in general) to adapt to user needs.

\subsection{Problem statement}

Modern campus buildings (e.g. Lab42) are increasingly equipped with sensors and automated systems to regulate occupants' comforts, governed by parameters established through generic building policies and faculty staff Unfortunately, students in these buildings often encounter limitations in exerting personal control over their comfort and data gathered by these systems is invisible to the user and choices the system makes are not transparent.

This adaptation implies that buildings should embody empathy \cite{empathic} and be adaptive (e.g. responsive to human signals such as emotions) with a focus on fostering user interaction with the environment. Researching to understand students' subjective needs, experiences, and behavior, coupled with a human-centric design approach, has the potential to elevate occupants' well-being and create optimal learning spaces. Furthermore, it provides valuable insights to faculty staff in making decisions to optimally arrange learning spaces and inform architecture and interior design studios on making decisions about structuring spaces and integrating computing technologies within built environments.

\subsection{Lab42 building}

This study will be conducted in association with the Digital Interactions Lab \footnote{https://uva-dilab.com/} and will utilize the recently opened Lab42 \footnote{https://lab42.uva.nl/} building at the UvA Amsterdam Science Park \footnote{https://www.amsterdamsciencepark.nl/} as its primary case study. Lab42 is an energy-neutral, flexible, and adaptable faculty building that facilitates collaborations among students, researchers, and businesses \cite{crouwel}. The buildings's layout is strategically organized into different zones, each serving various functions, ranging from quiet individual work to spaces that allow for collaborative work as shown in the building impression photographs in appendix \ref{appendix:building}. The overarching interior theme in the design revolves around 'tech' and 'nature' aiming to cultivate a fresh, light, and warm comfortable ambiance. Sensing devices are installed throughout the building to automatically adjust lighting, air, and temperature \cite{faculty}.