\section{Introduction}

This thesis will investigate [...] 

\subsection{Research questions}

In order to achieve this, the following main research question is formulated: \\

\emph{How do alterations in building activity, particularly adjustments in acoustic conditions and occupancy comfort, influence the levels of concentration among students and the establishment of conducive learning environments?} \\

To be able to answer this research question, the following supporting sub-questions are formulated:

\begin{itemize}
    \item \textbf{RQ1:} \emph{This is a research question.}
    \item \textbf{RQ2:} \emph{This is another research question.}
    \item \textbf{RQ3:} \emph{This is another research question.}
\end{itemize}

\subsection{Lab42 building}

This research will be performed in association with the \emph{Digital Interactions Lab} and uses the recently (september 2022) opened Lab42 \footnote{https://lab42.uva.nl/} building at the UvA Amsterdam Science Park as a case study. Lab42 is a energy-neutral, flexible and adaptable designed faculty building that facilites partnerships between students, researchers and businesses. \cite{crouwel} The layout aims to feature different zones with varying functionalities, from areas where you can sit quietly and focus on work to spaces that allows for collaborative work. The overarching interior theme in the design is 'tech' and 'nature' aiming to create afresh, light and warm comfortable building. Sensing devices are installed throughout the building to automatically adjust lighting, air, temperature so these can be adjusted for overall improvement of comfort \cite{faculty}.