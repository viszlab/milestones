\section{Introduction}
Globally, it is estimated that individuals typically allocate 90\% of their time within various indoor environments. New 'smart' buildings are often retrofitted with sensors and automated systems that aim to regulate comfort in a multi-dimensional approach encapsulated in thermal, respiratory, visual, and acoustic dimensions. 

One of these dimensions that can have far-reaching implications for occupants' comfort and health is the quality of indoor air (IAQ). Insufficient ventilation in indoor spaces leads to poor air quality and an increased occurrence of discomfort and diminished well-being experienced by occupants. The essence of air quality is its invisibility to occupants; polluted air is not easily detected by smell or sight. Additionally, mechanical ventilation systems in buildings operate discreetly, contributing to occupants' perceived lack of control. These systems are typically automated and cannot be directly regulated or controlled by occupants themselves.

This thesis focuses on understanding occupants' needs through in-the-wild studies measuring indoor air quality in specific spaces (e.g. meeting rooms) and prototyping various persuasive technologies and data physicalization devices to visualize indoor air quality and evaluating their effectiveness with the overall goal of gathering insights into occupants' comfort levels and helping them to take preventive action against poor indoor air quality. This creates an interplay between occupants' health and comfort, architecture and built environments, and computing technologies.



\subsection{Research questions}

In order to research intervention strategies for improving indoor air quality, the following main research question is formulated: \\

\emph{How can sensory measurements of air data be physically visualised in specific indoor spaces integrating both environmental information and elements that increase awareness among occupants facilitating their adoption of preventive measures against poor air quality?} \\

To effectively answer this main research question, this research is guided by the following supporting sub-questions that also serve as objectives to delineate the necessary knowledge:

\begin{itemize}
    \item \textbf{RQ1:} \emph{What are effective methods for collecting sensory measurements of air data in specific indoor spaces while ensuring accuracy and reliability?}
    \item \textbf{RQ2:} \emph{How can environmental information related to air quality, such as pollutant concentrations and ventilation rates, be incorporated into the visual representations?}
    \item \textbf{RQ3:} \emph{How do different types of physical visualizations impact occupants' understanding of air quality and their willingness to adopt preventive measures?}
    \item \textbf{RQ4:} \emph{How do occupants' perceptions and behaviors regarding indoor air quality change over time, from pre-installation to post-installation of the physical representation of poor air quality?}\\
\end{itemize}

As outlined in the \emph{related work} section existing research is performed on human-building interaction, the field of persuasive technology, data physicalization, and indoor air quality. While research on defining comfort within indoor buildings, gathering and analyzing sensory air data, and the effects of poor air quality are prevalent, there is a gap in understanding occupants' behavior and their subjective needs. Moreover, limited research exists on how design solutions visualizing environmental data and computing installations can empower occupants, providing them with control and taking preventive action.

\subsection{Problem statement}

Modern campus buildings are increasingly equipped with sensors and automated systems to regulate occupants' comforts, governed by parameters established through generic building policies, international standards, and faculty staff. Unfortunately, environmental data that is gathered such as indoor air quality and the parameters automated ventilation systems function upon are invisible to occupants in these buildings, and choices the system makes are not transparent. Occupants often encounter limitations in awareness of air quality and exerting personal control over their comfort.

Researching occupants' subjective needs, experiences, and behavior, coupled with a human-centric design approach, has the potential to improve occupants' well-being and create indoor environments with good indoor air quality. Furthermore, it provides valuable insights to faculty staff in making decisions in setting up ventilation systems, arranging indoor spaces, and informing architecture and interior design studios on making decisions about structuring spaces and integrating computing technologies within built environments.

\subsection{Lab42 building}

This study will be conducted in association with the Digital Interactions Lab \footnote{https://uva-dilab.com/} and will utilize the recently opened Lab42 \footnote{https://lab42.uva.nl/} building at the UvA Amsterdam Science Park \footnote{https://www.amsterdamsciencepark.nl/} as its primary case study. Lab42 is an energy-neutral, flexible, and adaptable faculty building that facilitates collaborations among students, researchers, and businesses \cite{crouwel}. The buildings's layout is strategically organized into different zones, each serving various functions, ranging from quiet individual work to spaces that allow for collaborative work as shown in the building impression photographs in appendix \ref{appendix:building}. The overarching interior theme in the design revolves around 'tech' and 'nature' aiming to cultivate a fresh, light, and warm comfortable ambiance. Sensing devices are installed throughout the building to automatically adjust lighting, air, and temperature \cite{faculty}.