\section{Related Work}

The desk research and literature review is related to several focus areas within human-computer interaction. A theoretical framework is first defined with the notion of human-building interaction and ubiquitous computing and then more specific to the use case case of this study related to persuasive technology and learning environments.

\subsection{Human-building Interaction}
Buildings increasingly incorporate new forms of interactivity, this means new inherent connections between 'people', 'built environments' and 'computing' in an emergent research area called Human-Building Interaction (HBI) \cite{hbi} and focussed on the design of built environments which may incorporate computing to varying degrees. Understanding how people use different spaces in a building can inform design interventions aimed at improving the utility of that building and inform the design of future buildings \cite{informed}. Currently research shows that much of the data collected by devices (e.g. sensors and cameras) are not immediately obvious to visitors and residents.

\subsection{Comfort within buildings}

Comfort is achieved in interaction with the environments and is represented in four respective dimensions; thermal, respiratory, visual and acoustic \cite{comfort}. Comfort can be studied and designed as interactive experience with the build environment \cite{environment}. Currently Indoor Environmental Quality (IEQ) indexes are being uses as a measurement of comfort and Post-Occupancy Evaluation (POE) to evaluate occupants perceived comfort. In current situations technology is typically retrofitted onto a new or existing building and users indicate a perceived lack of control and engagement with these systems since most are automated buildings set on arbitrarely set parameters. 

\subsection{Persuasive technology}

Ubiquitous computing (ubicomp) enables new forms and embodiment of computing, sensing and actuation combined with physicality with the goal of computing devices to 'dissapear' within the environment \cite{weiser}. Recent years have seen several ubicomp enables such as enabling lower-cost hardware (moore’s law), more diverse sensors and actuators and better protocol and communication technologies. Ubiquitous computing devices are often used as persuasive technology and designed to nudge people to change their behaviour \cite{twinkly}. These devices and technologies can extend the users awareness in a calm manor about impact of their decisions, through the use of the emerging notion of pervasive sensing \cite{calm}.

\subsection{Data physicalization}

The research field known as data physicalization has recently gained traction, it focusses on physical data visualizations making the invisible tangible and interactible by encoding data in physical artifacts \cite{tangible}. A shift from ‘artifact’ to ‘environment’ and thus enables a physical embodiment of computing. Data physicalization illustrates how opportunities, such as positively impacting how we perceive and explore data, compared to  'screen-focussed' (e.g. 2D canvas) data representations \cite{physicalization}.

\subsection{Informal Learning spaces}

Universities are increasingly re-arranging and re-building campus buildings to the notion of 'sticky campusses' \cite{sticky}, the notion to entice students to spend more time on campus. Primarily moving from strictly learning environments (e.g. lecture halls, classrooms) to more informal learning (e.g. collaborative spaces) spaces redefining universities as learning environments. The creation of these informal learning spaces, which are a large factor in the building of the Lab42 building, raises important questions regarding student behaviours and 'learning'\cite{critical} and there is increasing research focussed on the relation between learning spaces and student learning activities \cite{learning}