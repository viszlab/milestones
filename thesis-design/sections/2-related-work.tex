\section{Related Work}

The desk research and literature review span several key domains within the field of human-computer interaction. Initially, a theoretical framework is established with the notion of human-building interaction, comfort within buildings, and persuasive technology. Subsequently, the focus narrows down to more specific aspects relevant to the research context delving into data physicalization, and defining indoor air quality.

\subsection{Human-building Interaction}
Buildings increasingly incorporate new forms of interactivity, which means new inherent connections between 'people', 'built environments', and 'computing' in an emergent research area called Human-Building Interaction (HBI) \cite{hbi}. This research area is dedicated to exploring the design of built environments that may incorporate computing to varying degrees. Understanding how people use different spaces in a building can inform design interventions aimed at improving the utility of the space \cite{informed}. Current research highlights that a significant portion of the data collected by these computing devices within buildings is not readily apparent to visitors and residents, and buildings are not optimally architected to allow computing devices to be integrated within the environment.

\subsection{Comfort within buildings}

Comfort is achieved in interaction with the environment and is represented in four respective dimensions; thermal, respiratory, visual, and acoustic \cite{comfort}. Comfort can be studied and designed as an interactive experience with the built environment itself \cite{environment}. Indoor Environmental Quality (IEQ) indexes serve as metrics for assessing comfort, with Post-Occupancy Evaluation (POE) being employed to gauge occupants' perceived comfort. In current scenarios, technology is typically retrofitted onto a new or existing building and users indicate a perceived lack of control and engagement with these systems, primarily because many automated buildings operate based on arbitrarily set parameters, and data these sensors gather are often invisible to end users.

\subsection{Persuasive technology}

The primary aim of ubiquitous computing (ubicomp) devices is to seamlessly blend into the environment, essentially making computing devices 'disappear'\cite{weiser}. These devices are frequently employed as persuasive technology, strategically designed to gently nudge individuals towards behavior change leveraging the emerging notion of pervasive sensing to subtly enhance users' awareness regarding the impacts of their decisions \cite{twinkly}. This method of persuasive design serves as a powerful tool in calmly extending users' awareness, helping users understand the consequences of their actions, and gaining insight into their behavior \cite{calm}.

\subsection{Data physicalization}

The research domain known as data physicalization has emerged as a notable area of study, emphasizing the creation of physical data visualizations making the invisible tangible and interactible by encoding data in physical artifacts \cite{tangible}. This shift from focusing on individual artifacts to a broader environmental context facilitates the physical embodiment of computing. Data physicalization has the potential to positively influence the perception and exploration of data, presenting distinct advantages over traditional 'screen-focused' data representations, such as 2D canvas display \cite{physicalization}.

\subsection{Indoor Air Quality (IAQ)}
Globally, it is estimated that individuals typically allocate 90\% of their time within various indoor environments \cite{indoor}. Insufficient ventilation in indoor spaces leads to poor air quality and an increased occurrence of discomfort and diminished well-being experienced by occupants \cite{ventilation}. Poor air quality can have adverse impacts on health such as increased risk of asthma, headaches, and allergies \cite{who} as well as impact cognitive performance and productivity \cite{productivity}. Health Authorities worldwide recommend that one of the best solutions is to simply open the windows 2-3 times a day \cite{eu}. When occupants experience symptoms of sick building syndrome (SBS) \cite{sbs} the hazardous situation has already occurred. The goal is to recognize the role of humans as active users of a building and helping them prevent these situations before they occur.

\newpage