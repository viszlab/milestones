\section{Methodology}

Integration of user studies throughout the entire process, employing a systematic approach to data collection and prototyping design solutions are the main focusses of this research. The chosen methodologies are project-oriented and form part of in-the-wild studies aimed at examining user behavior and space utilization. These studies are complemented by data collected from sensing devices which all go through a processes of data cleaning, transformation, and analysis. As a culminating step, a prototype of persuasive technology will be developed and subjected to usability testing to assess potential changes in user behavior before and after installation. The overarching goal is to inform the design through data, utilizing design as a probing tool for data collection and vice versa. This approach ensures a comprehensive understanding of user interactions and space dynamics, ultimately contributing to the creation of a persuasive technology prototype that is both informed by empirical data and responsive to user needs.

\subsection{User studies (elicitation study)}

Collecting information about users within the Lab42 building will involve assessing their intentions within the buildings and their emotional states. Surveys are anticipated to be the primary data collection method distributed across the course of the thesis projects. Additionally, conducting one-on-one interviews using open-field questionnaires to explore users' comfort levels more comprehensively will also be used. The findings will be documented in methods such as personas, and empathy maps, and employing the MoSCoW method, providing an overview of user needs and current behavior within the building aiming to address the following questions:

\begin{itemize}
  \item What are the objectives and motivations of students upon entering the building?
  \item What specific learning objectives do students aim to achieve during their time in the building?
  \item What factors contribute to students choosing to work on campus instead of working from home?
\end{itemize}

\subsection{Space behavior (sensing devices)}

Analyzing current space behavior by utilizing the data generated from the automated systems already installed within the building. To augment this dataset, prototyped sensing devices may be introduced to collect specific sensory data not covered by the existing infrastructure. Concurrently, observations of students will be conducted to cross-reference the data with their actual behavior. The findings will be documented through methods such as field trials, customer journeys, and observation reports aiming to address the following questions:

\begin{itemize}
  \item What is the current distribution of space usage within the building?
  \item How frequently and for what duration do students typically spend time in the campus building?
  \item What criteria influence students' decisions in selecting a particular space for their work within the building?
\end{itemize}

\subsection{Prototyping (computing devices)}

After user research the aim is to formulate a design solution for behavior change, emphasizing the concept of calm technology—prompting users to undertake preventive actions with minimized interruption costs. The designed solution will undergo usability testing and subsequent data analysis, enabling a comparative evaluation of user behavior both before and after installation. The prototyping process is expected to comprise three interrelated components: \\

\textbf{1) Sensing device using a microcontroller: } development of a sensing device utilizing a microcontroller (e.g., ESP32 platform \footnote{\footnote{https://www.espressif.com/en/products/socs/esp32}}) to gather specific user behavior data in select spaces within the building. \\
\textbf{2) Real-time API-Integrated Storage: } creation of a storage system with a real-time API for persistent data storage in the backend. Visualizations for stakeholders will be displayed in a front-end dashboard for further analysis. This system is likely to be developed using frameworks such as Svelte \footnote{https://svelte.dev/} and the GraphQL query language \footnote{https://graphql.org/}.\\
\textbf{3) Tangible Data visualization: } crafting a physical, tangible data visualization presenting the collective output of the sensory data to influence behavior. This may involve utilizing platforms such as Raspberry Pi \footnote{https://www.raspberrypi.org/} or Arduino \footnote{https://www.arduino.cc/}, with possible visualization creation using Processing \footnote{https://processing.org/}.


\subsection{Datasets}
For the existing building used for this case study a dataset is provided which consists of "building data" to regulate automated systems, represented by an object list utilizing 'TAG encoding.' This list categorizes each installation and connection point, encompassing both generic elements like safety installations (e.g., smoke detectors), electrical components (e.g., wall sockets), and utilities (e.g., boilers), as well as sensory devices such as air quality regulators and light sensors controlling shutters. Additionally, the dataset includes floorplans and room number overviews.

Supplementary data comes from prior studies conducted on the Lab42 building. Master Student Jan Ramdohr conducted studies using air quality and light sensors for specific measurements in designated areas\cite{sensing}. A survey on user emotions across various spaces in the building was performed by Ph.D. candidate Shruti Rao \cite{emotion}.

Any additional data acquired from the user studies, space behavior observations, and prototype sensing devices will be processed (e.g. data cleaning, transformation, sentiment analysis) using Python \footnote{https://python.org/} and Jupyter \footnote{https://jupyter.org/}. Subsequently, the data will be subjected to analysis and visualization through graphs created using visualization libraries like Seaborn \footnote{https://seaborn.pydata.org/}.
