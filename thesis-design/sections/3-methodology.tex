\section{Methodology}

Integration of user studies throughout the entire process, employing a systematic approach to data collection, and prototyping design solutions are the main focuses of this research. The chosen methodologies are project-oriented and form part of in-the-wild studies aimed at examining occupant behavior. These studies are complemented by data collected from sensing devices which all go through a process of data cleaning, transformation, and analysis. Prototyping of sensing devices, persuasive technology, and data physicalizations will be developed and subjected to usability testing to assess potential changes in occupant behavior before and after installation. The overarching goal is to inform the design through data, utilizing design as a probing tool for data collection and vice versa. This approach ensures a comprehensive understanding of user interactions and space dynamics, ultimately contributing to a design solution that is both informed by empirical data and responsive to occupant needs.

\subsection{Space measurements (sensing devices)}

Analyzing current space with regards to air quality by utilizing the data measured from the automated systems already installed within the building exposed by the building API and augmenting this dataset with commercially available indoor climate data loggers from Atal \footnote{https://www.atal.nl/atu-ct-climatrend-binnenklimaat-datalogger}. Concurrently, observations of occupants will be conducted to cross-reference the data with their actual behavior. The findings will be documented through methods such as data frames, data processing, and data visualizations aiming to address the following questions:

\begin{itemize}
  \item What are the variations in indoor air quality currently within specific spaces over a defined period?
  \item When does indoor air quality reach suboptimal levels, and what are the thresholds indicating poor air quality?
  \item Can systems accurately predict periods of poor air quality within a specified timeframe?
\end{itemize}

\subsection{Occupant behavior (elicitation study)}

Collecting information about occupants within the specific spaces within the building will involve assessing their comfort states. Surveys are anticipated to be the primary data collection method distributed across the course of the thesis projects. Additionally, conducting one-on-one interviews using open-field questionnaires to explore occupants' comfort levels more comprehensively will also be used. The findings will be documented in methods such as personas, field trials, customer journeys, and observation reports aiming to address the following questions:

\begin{itemize}
  \item What is the level of awareness among occupants regarding indoor air quality and its influence on comfort?
  \item To what extent do occupants perceive and experience symptoms associated with poor indoor air quality?
\end{itemize}

\subsection{Prototyping (computing devices)}

After gathering data and some preliminary occupant studies the aim is to prototype a design solution for behavior change, emphasizing the concept of calm technology—prompting occupants to undertake preventive actions with minimized interruption costs. Most likely in the form of a tangible visualization of the air quality within a room. The designed solution will undergo usability testing and subsequent data analysis, enabling a comparative evaluation of user behavior both before and after installation. The prototyping process is expected to comprise three interrelated components: \\

\textbf{1) Sensing device using microcontrollers: } development of a sensing device utilizing a microcontroller (e.g., ESP32 platform \footnote{\footnote{https://www.espressif.com/en/products/socs/esp32}}) to gather specific user behavior data in select spaces within the building (e.g. sensors to detect if windows are opened). \\
\textbf{3) Tangible data visualization: } crafting a physical, tangible data visualization presenting the collective output of the sensory data to influence behavior (e.g. interactive installation installed on the walls that displays air quality). This may involve utilizing platforms such as Raspberry Pi \footnote{https://www.raspberrypi.org/} or Arduino \footnote{https://www.arduino.cc/}, with possible visualization creation using Processing \footnote{https://processing.org/}.


\subsection{Datasets}
For the existing building used for this case study a dataset is provided which consists of "building data" to regulate automated systems, represented by an object list utilizing 'TAG encoding.' This list categorizes each installation and connection point, encompassing both generic elements like safety installations (e.g., smoke detectors), electrical components (e.g., wall sockets), and utilities (e.g., boilers), as well as sensory devices such as air quality regulators and light sensors controlling shutters. Additionally, the dataset includes floorplans and room number overviews. 

This data will be supplemented by the aforementioned dataloggers installed in designated areas such as meeting rooms within the buildings for a certain period to gather air quality data. Supplementary data comes from prior studies conducted on the Lab42 building. Master Student Jan Ramdohr conducted studies using air quality and light sensors for specific measurements in designated areas\cite{sensing}. A survey on user emotions across various spaces in the building was performed by Ph.D. candidate Shruti Rao \cite{emotion}.

Any additional data acquired from the user studies, space behavior observations, and prototype sensing devices will be processed (e.g. data cleaning, transformation, sentiment analysis) using Python \footnote{https://python.org/} and Jupyter \footnote{https://jupyter.org/}. Subsequently, the data will be subjected to analysis and visualization through graphs created using visualization libraries like Seaborn \footnote{https://seaborn.pydata.org/}.
