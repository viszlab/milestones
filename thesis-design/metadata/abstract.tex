\begin{abstract}
In response to students' evolving demands, campus buildings are increasingly re-structured to serve as informal learning spaces integrating sensors and automated systems within learning environments. This research uses the Lab42 building at UvA Science Park as a case study to explore the impact on students of integrating ubiquitous computing for optimal learning activities. By analyzing student behavior within the building and their interactions with a prototype design solution, the findings provide valuable insights for faculty staff, offering potential implications for the design of future campus buildings and spaces.
\end{abstract}