\begin{abstract}
This master thesis investigates the integration of sensory measurements of indoor air quality (IAQ) into specific indoor spaces to raise occupants' awareness and help occupants take preventive measures against poor air quality. Conducted at Lab42 the study employs human-building interaction principles and persuasive technology to visualize IAQ data. Key research questions address methods for collecting accurate IAQ measurements, integrating environmental data into visual representations, and evaluating the impact of physical visualizations on occupants' understanding and behavior. The overall aim is to inform the development of effective interventions, contribute to healthier indoor environments and inform building design decisions.
\end{abstract}