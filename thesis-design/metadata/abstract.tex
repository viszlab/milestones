\begin{abstract}
While smart buildings incorporate sensors and automated systems to regulate various aspects of comfort, Indoor Air Quality remains a significant concern due to its invisible nature and its impact on occupants' health and well-being. This research uses the Lab42 building at UvA Science Park as a case study to study occupants' comfort through in-the-wild studies, focusing on specific spaces like meeting rooms and measuring the quality of air using sensory data and datalogger devices. Prototypes of persuasive technologies and data physicalizations are developed to make Indoor Air Quality visible and comprehensible to occupants enabling them to take preventive and proactive measures to mitigate poor air quality risks.
\end{abstract}